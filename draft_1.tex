\documentclass[11pt,a4paper]{article}

% Essential packages
\usepackage[utf8]{inputenc}
\usepackage{hyperref}
\usepackage{graphicx}
\usepackage{amsmath}
\usepackage{listings}
\usepackage{xcolor}
\usepackage{geometry}

% Document styling
\geometry{
    a4paper,
    margin=1in
}

% Title information
\title{\textbf{Open Captable Protocol (OCP):\\An Interopable Solution for Securities Settlement and Captable Management Onchain}}
\author{Victor Mimo}
\date{\today}

\begin{document}

\maketitle

\begin{abstract}
The Open Captable Protocol (OCP) introduces a revolutionary approach to managing capitalization tables through blockchain technology. This paper presents a comprehensive solution that bridges the gap between traditional security transfer systems and modern blockchain capabilities, while maintaining compliance and data privacy standards.
\end{abstract}

\section{Introduction}
\subsection{The Gap Between Web3 \& Securities Settlement}
\begin{itemize}
    \item Securities settlement remains primarily off-chain and fragmented, limiting interoperability.
    \item This is a problem because Web3 thrives on interoperability and permissionless (or sufficiently permissionless) actions.
    \item Wallets, DeFi, and protocols interoperate, yet captables remain siloed in Web2.
    \item Every crypto company today still uses Web2 databases to track cap tables.
    \item Even Coinbase explored going public on-chain but was laughed at (Coin documentary).
    \begin{itemize}
        \item This wasn't about captables---it was about the missing financial infrastructure (investment banks, liquidity, etc.).
    \end{itemize}
    \item On-chain captables seem unrealistic today---but why?
    \begin{itemize}
        \item Is it cost? Regulation? Lack of a standard?
    \end{itemize}
    \item Blockchains enable clearing, but settlement is still undefined.
    \item There is no standard on-chain framework for tracking securities ownership.
    \item Some have proposed hybrid approaches---such as Balaji's ``Mirrortables'' (2021), which mirrored Web2 cap tables on-chain.
    \begin{itemize}
        \item But mirroring isn't enough---it still treats legacy systems as primary.
        \item OCP flips this model: on-chain becomes the source of truth, and legacy systems can export into OCP---but OCP does not rely on them.
    \end{itemize}
    \item The missing piece isn't just technical---it's about adoption.
    \begin{itemize}
        \item To bring securities fully on-chain, we need to invest in this infrastructure today.
    \end{itemize}
    \item \textit{[Note: Settlement layers today: Tokenized securities exist on Base, Polygon, Solana, Ethereum---what's the right layer?]}
\end{itemize}

\subsection{What OCP Solves}
\begin{itemize}
    \item Built on Open Captable Format (OCF), a trusted and widely adopted standard in Web2.
    \item A fully on-chain, event-driven captable that is immutable, auditable, efficient, and scalable.
    \item Future-proofed to support more asset types as tokenized finance expands.
    \item Interoperable securities that protocols can integrate with natively for seamless ownership tracking and trading.
    \item Already in production, tracking over \$1B in private assets.
\end{itemize}

\section{Event-Based Smart Contract Captable}
\subsection{Common Misconceptions and Reality}
\begin{itemize}
    \item Captables are widely seen as a boring back-office necessity---whether they live in Excel, a database, or a SaaS platform, they ``just exist'' and don't add value.
    \item Many assume putting captables on-chain is unnecessary---they see it as over-engineering something that doesn't need innovation.
    \item People think captables ``get in the way'' of trading---a compliance burden that doesn't contribute to market efficiency.
    \item But an on-chain captable isn't just a better record-keeping tool---it creates a network of interoperable securities.
    \item Read \& write permissions are granted/removed with code, making securities management programmable and composable.
    \item This isn't just about tracking ownership---it's about making ownership actionable across protocols.
    \item People once thought fiat currency didn't need to be on-chain---but stablecoins changed that. OCP does the same for securities.
\end{itemize}

\section{OCP's Breakthrough}
\begin{itemize}
    \item \textbf{OCP is Fully Native \& Modular (Cross-Chain Compatible)}
    \begin{itemize}
        \item Unlike hybrid models that mirror Web2, OCP is fully native---legacy systems can export into OCP, but OCP does not rely on them.
        \item OCP's architecture is modular, meaning the core framework works across blockchains.
    \end{itemize}

    \item \textbf{Event-Driven Captables as the Core Model}
    \begin{itemize}
        \item Captables derive state from event-driven transactions, rather than mutable records.
        \item ActivePositions tracks ownership without external oracles or unnecessary disclosure.
        \item New facets (EVM) or programs (Solana) can be added dynamically to support evolving financial models.
    \end{itemize}

    \item \textbf{Blockchain-Specific Implementations}
    \begin{itemize}
        \item On EVM, we use the Diamond Architecture to enable scalable, gas-efficient captable growth.
        \item On Solana \& other chains, different scaling methods will be used, but the core OCP model remains the same.
        \item This ensures OCP is future-proof, allowing it to evolve as blockchain architectures improve.
    \end{itemize}
\end{itemize}

\section{Why This Is a Fundamental Shift}
\begin{itemize}
    \item \textbf{Securities as Interoperable Financial Primitives}
    \begin{itemize}
        \item In Web2, captables are static records that sit in private databases.
        \item In OCP, ownership can be used in smart contracts, DeFi, DAOs, and programmable markets.
        \item Just as stablecoins made fiat programmable, OCP makes securities programmable.
    \end{itemize}

    \item \textbf{From Compliance Burden to Programmable Governance}
    \begin{itemize}
        \item Traditionally, captables are compliance-first---designed for filings, not flexibility.
        \item With OCP, ownership is a live data layer, where permissions, governance, and transfers can be enforced at the smart contract level.
        \item This removes operational overhead while maintaining regulatory compliance.
    \end{itemize}

    \item \textbf{A New Layer for Capital Markets}
    \begin{itemize}
        \item Traditional finance is stuck with slow, fragmented infrastructure.
        \item OCP lays the foundation for real-time settlement, multi-chain securities, and global capital formation.
    \end{itemize}

    \item \textbf{No More Vendor Lock-In: Full Ownership Control}
    \begin{itemize}
        \item In Web2, switching cap table providers (e.g., Carta $\rightarrow$ Pulley) requires a full export and legal migration.
        \item With OCP, switching services doesn't require migration---just an ownership update on an already-deployed captable.
        \item Companies stay in control, eliminating data lock-in and making cap tables truly portable.
    \end{itemize}
\end{itemize}

\section{Interoperability}

\subsection{Two Types of Interoperability}

\textbf{1. Protocol-Level Interoperability:}
\begin{itemize}
    \item OCP isn't just a captable---it's a financial primitive that other protocols can build on.
    \item Other protocols (e.g., broker-dealers, issuance platforms, automated market makers) can become operators in OCP.
    \item RBAC (Role-Based Access Control) enables fine-grained permissioning, allowing external protocols to execute actions on behalf of an entity.
    \begin{itemize}
        \item Example: A solar security issuance protocol can let a trading platform handle secondary transactions without needing a full export of captable data.
        \item Example: A DeFi lending protocol can verify holdings in OCP to enable tokenized securities as collateral without requiring manual attestations.
    \end{itemize}
    \item This removes the bottlenecks of traditional financial reconciliation, where counterparties need manual data access---OCP makes securities live, programmable, and interoperable.
\end{itemize}

\textbf{2. Proof-of-Ownership with StakeholderNFT:}
\begin{itemize}
    \item OCP is extensible---protocols can build new layers on top of the captable.
    \item StakeholderNFT is one example of how ownership can be used beyond record-keeping.
    \item Investors can mint an NFT that dynamically reflects their holdings---proving ownership without exposing sensitive details.
    \begin{itemize}
        \item Privacy-Preserving: An investor can prove they hold shares in a company without revealing the exact number or valuation.
        \item Useful for governance, token-gated access, and reputation-based systems where full transparency isn't needed.
    \end{itemize}
    \item This enables governance, token-gated apps, identity verification, and DeFi collateralization.
    \item \textbf{Key Takeaway:} Ownership on OCP isn't just a static database entry---it's composable, verifiable, and programmable.
\end{itemize}

\section{Future Challenges and Long-term Vision}

\subsection{The Problem: Multi-Chain Market Contracts, Fragmented Settlement}
\begin{itemize}
    \item A solar field protocol deploys market contracts across multiple blockchains to maximize liquidity.
    \item Each chain records trades, but settlement needs to be unified.
    \item Right now, options might be:
    \begin{itemize}
        \item Deploying the captable on each chain (inefficient, complex).
        \item Using oracles or bridging solutions (potential trust assumptions).
    \end{itemize}
\end{itemize}

\subsection{The Vision: One Source of Truth for Settlement}
\begin{itemize}
    \item Whether a security is traded on Ethereum, Solana, Base, or another chain, there's one final authoritative record.
    \item The captable itself isn't fragmented---it syncs updates from multiple chains back to a unified source of truth.
\end{itemize}

\subsection{Three Possible Approaches to Achieving This Vision}

\subsubsection{1. Circle-Style API Model (Centralized Syncing \& On-Chain Settlement)}
\begin{itemize}
    \item A single entity (e.g., OCP Foundation) provides a cross-chain settlement API, similar to Circle's CCTP for USDC.
    \item Market contracts on multiple chains submit settlement events to a centralized service, which updates the captable.
    \item Smart contracts call the API to ensure on-chain finality, reducing fragmentation.
    \item \textbf{Challenges:}
    \begin{itemize}
        \item Requires a Web2 component (API calls).
        \item Introduces trust assumptions similar to how USDC relies on Circle.
    \end{itemize}
\end{itemize}

\subsubsection{2. On-Chain Cross-Chain Messaging (LayerZero/IBC-Style)}
\begin{itemize}
    \item Uses cross-chain messaging protocols like LayerZero, Axelar, or IBC to relay settlement updates between chains.
    \item Ownership is only updated when messages are verified by the captable contract on the primary chain.
    \item \textbf{Challenges:}
    \begin{itemize}
        \item Cross-chain bridges have security risks.
        \item Latency and finality differences between chains could cause syncing delays.
        \item Gas costs could be prohibitively high.
    \end{itemize}
\end{itemize}

\subsubsection{3. A Global Settlement Chain (Celestia-Style)}
\begin{itemize}
    \item Instead of choosing a specific blockchain like Ethereum or Solana, OCP could operate on a dedicated ``Settlement Layer.''
    \item Every trade on any chain must be reported to this single ledger, which serves as the canonical source of truth.
    \item Other chains would read from this chain when validating ownership.
    \item \textbf{Challenges:}
    \begin{itemize}
        \item Requires industry adoption---getting every protocol to agree on a single ledger is difficult.
        \item Gas fees---every security transaction needs to be committed to this chain.
    \end{itemize}
\end{itemize}

\subsection{Next Steps \& Research Areas}
\begin{itemize}
    \item Initially, OCP may explore the Circle-Style API model, as it is the most feasible short-term solution.
    \item Over time, cross-chain messaging or a dedicated settlement layer may provide a more trustless and decentralized approach.
    \item A key research focus is defining finality across multiple chains and determining how settlement syncs without fragmentation.
\end{itemize}

\section{Implementation and Adoption Strategy}
\subsection{OCP’s Open-Source Model}
\begin{itemize}
    \item Not a company, not a closed system—OCP is a free, open-source standard.
    \item Already in production, tracking over \$1B in private assets.
\end{itemize}

\subsection{What’s Next?}
\begin{itemize}
    \item Further adoption among issuers, broker-dealers, and DeFi protocols.
    \item Potential token model (TBD) to incentivize network effects.
\end{itemize}

\section{Conclusion}
\textbf{Call to Action:}
\begin{itemize}
    \item OCP is live—join the movement to bring securities fully on-chain.
\end{itemize}

\end{document}
